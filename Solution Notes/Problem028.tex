\documentclass{article}
\usepackage{amsmath,amssymb,listings}

\begin{document}
	
	\section*{Problem 28}
	
	Let \(k_n[r]\) be the corner numbers of the spiral, where \(n\in[0,3]\) are the four corners at a given radius \(r\in[0,\infty)\). We define \(k_n[0] = 1\). With this definition, the first few sequences are:
	
	\begin{align*}
		k[0] &= 1\\
		k[1] &= 3,5,7,9\\
		k[2] &= 13,17,21,25\\
		k[3] &= 31,37,43,49
	\end{align*}
	
	It is easy to recognize the pattern that forms and the sequence can be written more generally by:
	
	\begin{equation}
		k_n[r] = k_0[r] + 2nr\qquad\forall n\in [0,3],\quad r\in[1,\infty),\quad k_n[0] = 1
	\end{equation}
	
	The sequence of \(k_0[r]\) values is \(k_0[r] = 1,3,13,31,\dots\). We can write this sequence as the following:
	
	\begin{equation}
		k_0[r] = k_3[r-1] + 2r \qquad\forall r\in[1,\infty)
	\end{equation}
	
	Which gives us the general form of the discrete function for the corner numbers:
	
	\begin{equation}
		k_n[r] = k_3[r-1] + 2r +2rn\qquad\forall n\in[0,3],\quad r\in[1,\infty),\quad \text{with }k_n[0] = 1
	\end{equation}
	
	With this recurrant sequence definition, the sum of the diagonals of the spiral of radius 500 (1001x1001 square) is given by:
	
	\begin{equation}
		S = 1 + \sum_{r=1}^{500}\sum_{n=0}^{3}k_n[r]
	\end{equation}
	
	This is easy to program with recursion. An implementation in python follows.
	\newpage
	\begin{lstlisting}[language=Python]
def cornerNum(n,r):
	if not n in range(4):
		print "Invalid value of n"
		return None
	if r < 0:
		print "Invalid value of r"
		return None

	if r == 0:
		return 1

	else:
		return cornerNum(3,r-1) + 2*r + 2*r*n

sum = 1
for r in range(1,501):
	for n in range(0,4):
		sum += cornerNum(n,r)

print sum
	\end{lstlisting}
	
\end{document}